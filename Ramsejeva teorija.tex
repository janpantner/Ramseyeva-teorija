% Zadnja posodobitev: 14. 1. 2022
\documentclass[twoside,11pt]{article}
\usepackage[slovene]{babel}
\usepackage[utf8]{inputenc}
\usepackage{graphicx}
\usepackage[frame]{matrika}
\usepackage{mathtools}
\usepackage{epstopdf}
\usepackage{units}
% Po potrebi se lahk dodajo drugi standardni paketi, ki ne spreminjajo izgleda dokumenta

\usepackage{tikz}
\usepackage{amsmath}

\begin{document}

\MAT{1}{12}{2025}
\naslov{Ramseyeva teorija}

\avtor{Jan Pantner}

\institucija{Fakulteta za matematiko in fiziko \\ Univerza v Ljubljani}

\klasifikacija{05D10} 
\izvlecek{Povzetek v slovenščini}
\title{Ramsey theory}
\abstract{Povzetek v angleščini.}

\glava\baselineskip=14.5pt

\smallskip

\section{Uvod}

Ramseyeva teorija govori o particijah velikih struktur. Tipičen rezultat nam pove, da 
se mora v neki particiji dovolj velike strukture pojaviti neka specifična podstruktura. Na primer, 
če povezave dovolj velikega polnega grafa pobarvamo z nekaj barvami, mora nujno obstajati 
monokromatičen trikotnik (v tem primeru imamo particijo povezav na različne barve, 
osnovna struktura so povezave grafa, iskana podstruktura pa monokromatičen trikotnik).

Da zgornje res drži, nam zagotovi Ramseyev izrek, ki je nekakšna ``večdimenzionalna'' posplošitev 
Dirichletovega principa, ki tudi sam preučuje particije struktur (na primer razdelitev golobov v
golobnjake).

Bolj filozofsko, Ramseyeva teorija pove da ne gleda na to, kako kaotičen je nek sistem, 
bodo znotraj sistema vedno obstajali urejeni deli.

\section{Osnovni rezultati}
\begin{zgled}
    Ali med poljubnimi šestimi ljudmi vedno obstajajo trije, ki se med seboj
    poznajo, ali trije, ki se med seboj ne poznajo?

    Poglejmo enega izmed teh ljudi. Brez škode za splošnost po Dirichletovem principu med 
    ostalimi petimi obstajajo trije, ki jih pozna. Če se poljubna dva med temi tremi poznata, 
    imamo tri ljudi, ki se med seboj poznajo. Sicer se ti trije med seboj ne poznajo, odgovor je 
    torej v obeh primerih pritrdilen.
\end{zgled}

Zgornji zgled je eden izmed enostavnejših rezultatov Ramseyeve teorije, vendar v nadaljevanju,
za lažje razumevanje, izjave raje formuliramo v jeziku teorije grafov in ne socialne interakcije.

Kot zanimivost lahko omenimo, da se Ramsey v 
resnici ni ukvarjal niti z grafi niti s socialnimi interakcijami, temveč z logiko. Izrek, po katerem 
je najbolj znan, je dokazal kot manjšo lemo na poti k popolnoma drugačnim rezultatom. Večina zgodnjega razvoja 
Ramseyeve teorije se je zgodila šele po njegovi smrti pri ranih $26$-tih letih. Za zgodnji razvoj je 
med drugimi v veliki meri zaslužen eden najznamenitejših matematikov 20.~stoletja 
Paul~Erdős.

\begin{izrek}[Ramsey]
    Za poljubni naravni števili $r$ in $s$ obstaja najmanjše takšno naravno število 
    $n = R(r,s)$, da velja naslednje: Če povezave polnega grafa $K_n$ pobarvamo z dvema barvama, 
    zagotovo obstaja poln podgraf moči $r$, v katerem so vse povezave prve barve ali pa poln 
    podgraf moči $s$, v katerem so vse povezave druge barve.
\end{izrek}

Številu $R(r, s)$ pravimo \emph{Ramseyevo število}.
Alternativna formulacija zgornjega izreka bi bila, da vsak graf na $n$ točkah vsebuje 
bodisi kliko moči $r$ bodisi antikliko moči $s$.
Dokaz te verzije izreka izpustimo, izrek !!!! predstavlja splošnejšo verzijo.

\begin{zgled}
    Določimo $R(3,3)$ in $R(3, 4)$.

    Pokazali smo že, da velja $R(3, 3) \le 6$ (ljudi predstavimo z vozlišči, poznanstva pa z barvo povezave). Za dokaz spodnje meje si poglejmo 
    sliko \ref{fig:r33}.
    \begin{figure}[h!]
        \centering
        \newcommand\size{1}
        \begin{tikzpicture}[scale=2]
            \draw[thick,blue]  (18:\size) \foreach \a in {90,162,234,306} { -- (\a:\size) } -- cycle;
            \draw[thick,red] (18:\size) \foreach \a in {162,306,90,234} { -- (\a:\size) } -- cycle;
            \foreach \a in {18,90,162,234,306} { \node[black,fill=black,circle,inner sep=1.5pt] at (\a:\size){}; }
        \end{tikzpicture}
        \caption{Barvanje $K_{5}$ z dvema barvama.}
        \label{fig:r33}
    \end{figure}
    Opazimo, da barvanje nima monokromatičnega trikotnika, torej $R(3, 3) > 5$. Skupaj smo dokazali 
    $R(3,3) = 6$.

    Slika \ref{fig:r34} prikazuje barvanje $K_{8}$ z rdečo in modro barvo, 
    ki nima rdečih trikotnikov in nima modrih klik velikosti $4$. Sledi $R(3, 4) > 8$. 
    \begin{figure}[h!]
        \centering
        \newcommand\size{1}
        \begin{tikzpicture}[scale=2]
            \draw[thick,blue]  (0:\size) \foreach \a in {0,45,90,135,180,225,270,315} { -- (\a:\size) } -- cycle;
            \draw[thick,blue]  (0:\size) \foreach \a in {0,90,180,270} { -- (\a:\size) } -- cycle;
            \draw[thick,blue]  (45:\size) \foreach \a in {45,135,225,315} { -- (\a:\size) } -- cycle;
            \draw[thick,red] (0:\size) \foreach \a in {0,135,270,45,180,315,90,225} { -- (\a:\size) } -- cycle;
            \draw[thick,red] (0:\size) \foreach \a in {0,180} { -- (\a:\size) } -- cycle;
            \draw[thick,red] (45:\size) \foreach \a in {45,225} { -- (\a:\size) } -- cycle;
            \draw[thick,red] (90:\size) \foreach \a in {90,270} { -- (\a:\size) } -- cycle;
            \draw[thick,red] (135:\size) \foreach \a in {135,315} { -- (\a:\size) } -- cycle;
            \foreach \a in {0,45,90,135,180,225,270,315} { \node[black,fill=black,circle,inner sep=1.5pt] at (\a:\size){}; }
        \end{tikzpicture}

        \caption{Barvanje $K_{8}$ z dvema barvama.}
        \label{fig:r34}
    \end{figure}

    Pokažimo, da velja $R(3, 4) = 9$. Recimo, da imamo barvanje $K_9$ z rdečo in modro 
    barvo, ki ne vsebuje rdečega trikotnika in ne vsebuje modre klike moči $4$. Poglejmo 
    si neko vozlišče $v_0$. 
    \begin{itemize}
        \item Recimo, da je vozlišče $v_0$ vsebovano v vsaj šestih modrih povezavah. Zaradi 
        prejšnjega dela naloge lahko med temi šestimi vozlišči najdemo tri vozlišča,
        ki tvorijo moder trikotnik. Skupaj z vozliščem $v_0$ tedaj tvorijo modro kliko moči 
        $4$.

        \item Recimo, da je vozlišče $v_0$ vsebovano v vsaj štirih rdečih povezavah. Če 
        med temi štirimi vozlišči (ki so z $v_0$ povezana z rdečo povezavo) ni nobene 
        rdeče povezave, tvorijo modro kliko moči $4$. Če obstajata med njimi vozlišči, 
        ki sta povezani z rdečo povezavo, tedaj skupaj z $v_0$ tvorita rdeč trikotnik.
    \end{itemize}
    
    Edina preostala možnost je, da je vsako vozlišče vsebovano v natanko treh rdečih povezavah. 
    V tem primeru podgraf, vpet na rdečih povezavah, krši lemo o rokovanju.
\end{zgled}

\section{Dokaz Ramseyevega izreka}

\begin{izrek}[Ramsey]
    Naj bo $r \ge 1$ in $a_1, a_2 \ge r$. Tedaj obstaja najmanjše takšno naravno število 
    $N(a_1, a_2; r)$, da velja naslednje: Če v množici $S$ moči $n \ge N(a_1, a_2; r)$
    vse $r$-podmnožice pobarvamo z barvo $1$ ali $2$, potem obstaja takšna $a_1$-podmnožica, 
    da so vse njene $r$-podmnožice barve $1$, ali pa obstaja takšna $a_2$-podmnožica, da so 
    vse njene $r$-podmnožice barve $2$. \label{ramsey}
\end{izrek}

\begin{dokaz}
    Uporabimo dvojno indukcijo, po $r$ in še po $a_1 + a_2$. 
    V primeru $r = 1$ velja $N(a_1, a_2; 1) = a_1 + a_2 -1$. Recimo, da je $a_1 \ge a_2$. Potem je 
    $N(a_1, a_2; a_2) = a_1$. Sedaj predpostavimo, da izrek velja za $r-1$ in ga dokažimo za $r$.
    Naj bo
    \begin{align*}
        a_1' &:= N(a_1 - 1, a_2; r), \\
        a_2' &:= N(a_1, a_2 -1; 4).
    \end{align*}
    %Naj bo $S$ množica moči več kot $N(a_1', a_2'; r-1) + 1$. Vse podmnožice $S$ pobarvamo z 
    %barvo $1$ ali barvo $2$. Naj bo $a \in S$ in $S' := S \setminus \set{a}$. Barvanje $S'$ je 
    %usklajeno z barvanjem $S$ tako, da je barva $X \subseteq S'$ enaka barvi $X \cap \set{a}$ v $S$.

    %Ker velja $\abs{S'} \ge N(a_1', a_2';r-1)$, brez škode za splošnost obstaja $a_1'$-podmnožica $A$ 
    %množice $S'$, v kateri so vse $(r-1)$-podmnožice barve $1$. Velja $\abs{A} = a' = N(a_1-1,a_2;r)$.
    %Sedaj ločimo dva primera.
    %\begin{itemize}
    %    \item Recimo, da v $A$ obstaja $a_2$-podmnožica, v kateri so vse $r$-podmnožice barve $2$.
    %    \item Recimo, da v $A$ obstaja $(a_1-1)$-podmnožica $A'$, v kateri so vse $r$-podmnožice barve $1$. 
    %    Tedaj definiramo
    %    \[
    %        A'' := A' \cap \set{a}.
    %    \]
    %    Velja $\abs{A''} = a_1$. Ker je barvanje $S' \supseteq A'$ usklajeno z barvanjem $S \supseteq A$, 
    %    so vse $r$ podmnožice v $A''$ barve $1$.
    %\end{itemize}
    V obeh primerih smo izpolnili zahtevan pogoj, torej izrek velja tudi za $r$. \hfill $\blacksquare$
\end{dokaz}

Naslednja trditev je zanimiv geometrijski rezultat, katerega dokaz 
temelji na posplošeni verziji Ramseyevega izreka. 

\begin{trditev}[Erdős-Szekeres 1935]
    Za vsak $n \in \mathbb{N}$ obstaja tako število $N(n)$, da velja: Če imamo v ravnini $N \ge N(n)$
    točk v splošni legi, potem med njimi obstaja $n$ točk, ki določajo konveksen $n$-kotnik.
\end{trditev}

Najprej dokažimo naslednjo geometrijsko lemo.
\begin{lema}
    Množica $n$ točk v ravnini tvori konveksen $n$-kotnik natanko tedaj, kadar
    vsaka podmnožica štirih točk tvori konveksen $4$-kotnik.
\end{lema}
\begin{dokaz}
    Poglejmo konveksno ogrinjačo točk. Če točke tvorijo konveksen $n$-kotnik, potem vsake 
    štiri tvorijo konveksen štirikotnik. Nasprotno, recimo, da točke ne tvorijo konveksnega 
    $n$-kotnika. Potem obstaja točka znotraj ogrinjače. Če ogrinjačo trianguliramo, bo ta 
    točka znotraj nekega trikotnika in skupaj z oglišči trikotnika ne bo tvorila 
    konveksnega $4$-kotnika. \hfill $\blacksquare$
\end{dokaz}

\noindent\textit{Dokaz trditve.}
    Naj bo $N \ge N(n) := R(n,n;3)$. Točke označimo z $1, 2, \dots, N$. 
    Poglejmo poljuben trikotnik in njegova oglišča označimo z $i$, $j$, $k$ tako, da $i < j < k$. Če je 
    trikotnik $IJK$ pozitivno orientiran, ga pobarvamo s prvo barvo, sicer z drugo.

    Ker je $N \ge N(n,n;3)$, brez škode za splošnost obstaja $n$-podmnožica, v kateri so 
    vsi trikotniki prve barve. Dokažimo, da ta množica tvori konveksen $n$-kotnik. Predpostavimo 
    nasportno in uporabimo lemo. Torej obstajajo štiri točke, ki ne tvorijo konveksnega 
    štirikotnika. Če pogledamo orientacije trikotnikov na teh štirih vozliščih, pridemo do 
    protislovja s tem, da so vsi trikotniki iste barve. \hfill $\blacksquare$

\section{O računski zahtevnosti} 

Iskanje točnih vrednosti Ramseyevih števil hitro postane zelo zahtevno in nedosegljivo današnji 
tehnologiji. Tabela \ref{tab:meje} prikazuje trenutne znane vrednosti in meje za Ramseyeva števila
$R(r, s)$. 

\begin{table}[h!]%[H]
    \centering
    \begin{tabular}{|c||c|c|c|c|c|c|c|c|}\hline
    $R(r,s)$ & 3 & 4  & 5     & 6     & 7     & 8     & 9       & 10     \\\hline \hline
    3 & 6 & 9  & 14    & 18    & 23    & 28    & 36      & 40-41  \\\hline
    4 &   & 18 & 25    & 36-40 & 49-58 & 59-79 & 73-105 & 92-135 \\\hline
    5 &   &    & 43-46 & 59-85 & 80-133 & 101-193 & 133-282 & 149-381\\\hline
    \end{tabular}
    \caption{Znane vrednosti/meje za Ramseyeva števila $R(r, s)$.}
    \label{tab:meje}
\end{table}
Najnovejša sprememba v tabeli se je zgodila decembra leta $2023$, ko je bil objavljen 
članek, v katerem je pokazano $R(3, 10) \le 41$. 
O tem rezultatu govori \cite{r41}.

Vidimo, da je v resnici znanih zelo malo Ramseyevih števil. Če želimo dokazati, 
da je $R(s, t) = N$, moramo najti ustrezno barvanje povezav grafa $K_{N-1}$ in pokazati, da 
vsa barvanja povezav grafa $K_{N}$ ustrezajo nekemu pogoju. V teoriji bi lahko uporabili računalnik 
in preverili vsa možna barvanja za zaporedne $n$, dokler ne najdemo prvega $N$, pri katerem vsako 
barvanje zadošča zahtevanemu pogoju. Že v primeru dveh barv postane število barvanj zelo hitro 
nepredstavljivo veliko. V primeru, ko imamo več kot dve barvi ali pa v primeru $R(s, t; r)$, kjer $r > 2$, je znanega še
veliko manj.

\section{Zgornje in spodnje meje}

V tem razdelku dokažemo nekaj zgornjih in spodnjih mej za Ramseyeva števila.

\begin{trditev}
    Naj bosta $a_1,a_2 \ge 2$. Velja
    \[
        N(a_1, a_2; 2) \le \binom{a_1 + a_2 -2}{a_1-1}.
    \]
\end{trditev}

\begin{dokaz}
    Uporabimo indukcijo. Velja
    \[
        N(a_1, 2; 2) = a_1 = \binom{a_1+2-2}{a_1-1}
    \]
    in podobno za $N(2,a_2; 2)$. Direktno iz dokaza 
    Ramseyevega izreka sledi
    \begin{equation}
        R(a_1, a_2; r) \le N(N(a_1-1,a_2;r), N(a_1,a_2-1;r);r-1) + 1.
        \label{eq:meja}
    \end{equation}
    Za dokaz indukcijskega koraka nazadnje upoštevamo še $N(a_1,a_2;1) = a_1 + a_2 - 1$ in dobimo
    \begin{align*}
        N(a_1, a_2; 2) &\overset{\eqref{eq:meja}}{\le} N(N(a_1-1,a_2;2), N(a_1,a_2-1;2);1) +1\\
        &= (N(a_1-1,a_2;2) + N(a_1,a_2-1; 2) - 1) + 1 \\
        &\overset{{\text{IP}}}{\le} \binom{a_1 + a_2 -3}{a_1-2} + \binom{a_1 + a_2 -3}{a_1-1} \\
        &=\binom{a_1+a_2-2}{a_1-1},
    \end{align*}
    kjer zadnja enakost velja zaradi rekurzivne formule za binomski koeficient. \hfill $\blacksquare$
\end{dokaz}

%%%% ERDOS 


\begin{izrek}[Erdős]
    
\end{izrek}

\begin{dokaz}
    \hfill $\blacksquare$
\end{dokaz}

% Dolga leta bla bla bla --- > izboljšava??


% Stvari iz zapiskov predavanja na GimB
% Erdosova probabilistična spodnja meja
\begin{thebibliography}{99}
    \bibitem{r41} V.~Angeltveit, \emph{R(3,10) <= 41}, 2024, arXiv: 2401.00392 [math.CO].
    \bibitem{west} D.~B.~West, \emph{Introduction to Graph Theory}, Prentice Hall, 2001.
    \bibitem{citekey} \emph{Ramsey's theorem}, Wipedia, https://en.wikipedia.org/wiki/Ramsey's\_theorem, Pridobljeno: 28.~1.~2025
    
%\bibitem{7} E.A. Cornell, C.E. Wieman, M.R. Matthews, J.R. Ensher in M.H. Anderson, \emph{Observation of Bose-Einstein Condensation in a Dilute Atomic Vapor}, Science \textbf{269} (1995), 198--201. 
%\bibitem{6} W. Ketterle, D.M. Kurn, D.S. Durfee, N.J. van Druten, M.R. Andrews, M.-O. Mewes in K.B. Davis, \emph{Bose-Einstein Condensation in a Gas of Sodium Atoms}, Physics Review Letters \textbf{75} (1995), 3969--3973. 
%\bibitem{3} J. Klaers, J. Schmitt, F. Vewinger in M. Weitz, \emph{Bose-Einstein condensation of photons in an optical microcavity}, Nature \textbf{468} (2010), 545--548. 
%\bibitem{10} T. Giamarchi, C. R\"uegg in O. Tchernyshyov, \emph{Bose-Einstein condensation in magnetic insulators}, Nature Physics \textbf{4} (2008), 198--204. 
%\bibitem{1} J.F. Annett, \emph{Superconductivity, Superfluids and Condensates}, Oxford University Press, 2004.
%\bibitem{8} L. Lewin, \emph{Polylogarithms and Associated Functions}, Elsevier Science Ltd., 1981. 
%\bibitem{9} H. Wagner in N.D. Mermin, \emph{Absence of Ferromagnetism or Antiferromagnetism in One- or Two-Dimensional Isotropic Heisenberg Models}, Physics Review Letters  \textbf{17} (1996), 1133--1136.
%\bibitem{2} J. Klaers, F. Vewinger in M. Weitz, \emph{Thermalization of a two-dimensional photonic gas in a ‘white-wall’ photon box}, Nature Physics \textbf{6} (2010), 512--515.
\end{thebibliography}

\end{document}
